\documentclass[11pt, ngerman, fleqn, DIV=15, headinclude]{scrartcl}

\usepackage[bibatend, color]{../header}

\hypersetup{
    pdftitle=
}

\renewcommand{\thesubsection}{\thesection.\alph{subsection}}

%\subject{}
\title{Statistische Physik, Blatt 1}
%\subtitle{}
\author{
    Frederike Schrödel \and Jan Weber \and Simon Schlepphorst
}

\usepackage{mathtools}

\renewcommand{\thesubsection}{\thesection.\arabic{subsection}}
\renewcommand{\thesubsubsection}{\thesubsection.\alph{subsubsection}}

\usepackage{cancel}

\begin{document}

\maketitle
\begin{center}
	\begin{tabular}{l|c|c|c}
		Aufgabe &1&2&$\Sigma$\\
		\hline
		Punkte &\quad /10 & \quad /15 & \quad /25 
	\end{tabular}\\
\end{center}


\setcounter{section}{1}

\subsection{Maxwell Relationen und Ableitungsregeln}


\subsection{Wahrscheinlichkeitstheorie}

\subsubsection{}
Es soll eine mögliche Zufallsvariable $X$ für das Würfeln mit 2 Würfeln angegeben werden.\\
Ich nehme: $X=$ Summe der Augen ist 7.

\subsubsection{}
Nun soll angegeben werden, wie groß die Wahrscheinlichkeit ist, dass die in a. definierte Zufallsvarianble $X$ auftritt.\\
Eine 7 beim Würfeln mit zwei Würfeln erhält man bei den sechs Kombinationen:
\{$\del{1,6}$, $\del{6,1}$, $\del{2,5}$, $\del{5,2}$, $\del{3,4}$, $\del{4,3}$\} von insgesamt 36 möglichen, so dass folgt:
\begin{align*}
	P_X\del{X=7}=\frac6{36}=\frac16
\end{align*}

\subsubsection{}
Als nächstes soll der Mittelwert aller möglichen Ergebnisse des Würfelns mit zwei Würfeln berechnet werden. Da die Wahrscheinlichkeiten $P_{X_i}, i =\{2,3,...,12\}$ symmetrisch um die 7 verteilt sind ($P_{X_2} = P_{X_{12}}; P_{X_3} = P_{X_{11}}; ...$ ist der Mittelwert offensichtlich 7.

\subsubsection{}
Nun soll gezeigt werden, dass für das Schwankungsquadrat
\begin{align*}
	\del{\Delta X^2}\equiv\braket{\del{X - \braket{X}}^2}
\end{align*}
folgt:
\begin{align*}
	\del{\Delta X^2}=\braket{X^2} - \braket{X}^2
\end{align*}
Dazu benötige ich, dass der Erwartungswert des Erwartungswertes der Erwartungswert ist, also $\braket{\braket{X}} = \braket{X}$. Dann folgt:
\begin{align*}
	\del{\Delta X^2}	&= \braket{\del{X - \braket{X}}^2}\\
						&= \braket{X^2 - 2X\braket{X} + \braket{X}^2}\\
						&= \braket{X^2} - 2\braket{X\braket{X}} + \braket{X}^2\\
						&= \braket{X^2} - 2\braket{X}\braket{X} + \braket{X}^2\\
						&= \braket{X^2} - 2\braket{X}^2 + \braket{X}^2\\
						&= \braket{X^2} - \braket{X}^2
\end{align*}

\subsubsection{}
Nun wird der Korrelationskoeffizient $K_{ij}$ eingeführt mit:
\begin{align*}
	K_{ij} = \braket{\del{X_i - \braket{X_i}}\del{X_j - \braket{X_j}}}
\end{align*}
Es werden zwei Zufallsexperimente betrachtet:
\begin{enumerate}
	\item
	Das Würfeln zweier Würfel, wobei die erste Zufallsvariable der erste Würfel und die zweite Zufallsvariable der zweite Würfel ist.\\
	In diesem Fall ist der Mittelwert beider Zufallsexperimente 3,5. Somit ist dann die Differenz der Werte zum Mittelwert: $X_i - \braket{X_i} = \{-2,5;-1,5;-0,5;0,5;1,5;2,5\}$.\\
	Betrachtet man also den Mittelwert aller Koeffizienten heben sich die Positiven und negativen Werte gegeneinander weg und es folgt sofort:
	\begin{align*}
		K_{ij} = 0
	\end{align*}
	\item
	Als zweites schießt ein Physiker zwei mal auf ein Tor, dabei trifft er beim ersten mal zu 50\%, beim zweiten mal zu 50\% wenn er beim ersten Schuß getroffen, zu 25\%, wenn er beim ersten mal nicht getroffen hat.\\
	Die Wahrscheinlichkeit beim zweiten mal zu treffen hängt also von dem Ergebnis des ersten Experiments ab. Im folgenden bezeichne ich einen Treffer als positives Ereignis (+), einen Fehlschuss als negativ (-).\\
	Die Wahrscheinlichkeiten sind:
	\begin{align*}
		&\braket{X_2^1\del{+}} = 0,5\\
		&\braket{X_2^1\del{-}} = 0,5\\
		&\braket{X_2^2\del{+}} = 0,5\\
		&\braket{X_2^2\del{-}} = 0,25
	\end{align*}
	Damit folgt für den Korelationskoeffizienten:
	\begin{align*}
		K_{ij} 	&= \frac14\del{\cancel{\del{1 - 0,5}\del{1 - 0,5}} + \cancel{\del{1 - 0,5}\del{0 - 0,5}} + \del{0 - 0,5}\del{1 - 0,25} + \del{0 - 0,5}\del{0 - 0,5}}\\
				&= \frac14\del{\del{0 - 0,5}\del{1 - 0,25} + \del{0 - 0,5}\del{0 - 0,5}}\\
				&= \frac14\del{-0,5\cdot0,75 + -0,5\cdot-0,5}\\
				&= \frac14\del{-0,375 + 0,125}\\
				&= \frac14\cdot-0,25\\
				&= -0,0625
	\end{align*}
\end{enumerate}

\subsubsection{}
In diesem Aufgabenteil soll ein Dopingtest betrachtet werden, der bei Sportveranstaltungen durchgeführt wird.\\
Es werden drei Fälle betrachtet:
\begin{enumerate}
	\item
	Wie wahrscheinlich es ist, dass eine Dopingprobe positiv ist:
	\begin{align*}
		P_1 = 0,99\cdot0,2 + 0,05\cdot0,8 = 0,238
	\end{align*}
	\item
	Wie wahrscheinlich es ist, dass ein Test positiv ausfällt, obwohl der Sportler gedopt hat:
	\begin{align*}
		P_2 = 0,2*0,01 = 0,002
	\end{align*}
	\item
	Wie wahrscheinlich es ist, dass ein Sportler gedopt hat, falls seine Dopingprobe negativ gewesen ist:
	\begin{align*}
		P_3 = \del{1 - P_1}P_2 = 0,001524
	\end{align*}
\end{enumerate}

\subsubsection{}
Zuletzt werden Bernoulli-Prozesse betrachtet. Die Wahrscheinlichkeit ist dann durch die Beinominalverteilung
\begin{align*}
	B\del{p,k} = 
	\begin{pmatrix}
		N\\
		k			
	\end{pmatrix}
	p^k\del{1-p}^{N-k}
\end{align*}
gegeben.\\
Es sollen Erwartungswert und Schwankungsbreite ausgerechnet werden.\\
Zuerst der Erwartungswert:
\begin{align*}
	\braket{k}	&= \sum_{k=0}^N k
				\begin{pmatrix}
					N\\
					k			
				\end{pmatrix}
				p^k\del{1-p}^{N-k}\\
				&= \sum_{k=0}^N \frac{kN!}{k!\del{N-k}!}p^k\del{1-p}^{N-k}\\
				&= \sum_{k=1}^N \frac{N!}{\del{k-1}!\del{N-k}!}p^{k}\del{1-p}^{N-k}\\
	\intertext{%
		Setze $k' \equiv k - 1 \Leftrightarrow k = k' + 1$
	}
				&= \sum_{k'=0}^{N-1} \frac{N!}{k'!\del{N-1-k'}!}p^{k'+1}\del{1-p}^{N-1-k'}\\
				&= Np\sum_{k'=0}^{N-1} \frac{\del{N-1}!}{k'!\del{N-1-k'}!}p^{k'}\del{1-p}^{N-1-k'}\\
				&= Np\underbrace{\sum_{k'=0}^{N-1}
				\begin{pmatrix}
					N-1\\
					k'			
				\end{pmatrix}
				p^{k'}\del{1-p}^{N-1-k'}}_{=1}\\
				&= Np
\end{align*}
Jetzt den Erwartungswert des Quadrats:
\begin{align*}
	\braket{k^2}&= \sum_{k=0}^N k^2
				\begin{pmatrix}
					N\\
					k			
				\end{pmatrix}
				p^k\del{1-p}^{N-k}\\
				&= \sum_{k=0}^N \del{k\del{k-1} + k}
				\begin{pmatrix}
					N\\
					k			
				\end{pmatrix}
				p^k\del{1-p}^{N-k}\\
				&= \sum_{k=0}^N k\del{k-1}
				\begin{pmatrix}
					N\\
					k			
				\end{pmatrix}
				p^k\del{1-p}^{N-k} + \underbrace{\sum_{k=0}^N k
				\begin{pmatrix}
					N\\
					k			
				\end{pmatrix}
				p^k\del{1-p}^{N-k}}_{=\braket{k}}\\
				&= \sum_{k=0}^N \frac{k\del{k-1}N!}{k!\del{N-k}!}p^k\del{1-p}^{N-k} + \braket{k}\\
				&= \sum_{k=2}^N \frac{N!}{\del{k-2}!\del{N-k}!}p^{k}\del{1-p}^{N-k} + \braket{k}\\
	\intertext{%
		Setze $k' \equiv k - 2 \Leftrightarrow k = k' + 2$
	}
				&= \sum_{k'=0}^{N-2}\frac{N!}{k'!\del{N-2-k'}!}p^{k'+2}\del{1-p}^{N-2-k'} + \braket{k}\\
				&= N\del{N-1}p^2\sum_{k'=0}^{N-2}\frac{\del{N-2}!}{k'!\del{N-2-k'}!}p^{k'}\del{1-p}^{N-2-k'} + \braket{k}\\
				&= N\del{N-1}p^2\underbrace{\sum_{k'=0}^{N-2}
				\begin{pmatrix}
					N-2\\
					k'		
				\end{pmatrix}
				p^{k'}\del{1-p}^{N-2-k'}}_{=1} + \braket{k}\\
				&= N\del{N-1}p^2 + \braket{k}\\
				&= N\del{N-1}p^2 + Np
\end{align*}
Es folgt:
\begin{align*}
	\braket{k^2} - \braket{k}^2	&= N\del{N-1}p^2 + Np - N^2p^2\\
								&= N^2p^2 - Np^2 + Np - N^2p^2\\
								&= Np - Np^2\\
								&= Np\del{1 - p}
\end{align*}














\end{document}

% vim: spell spelllang=de tw=79
