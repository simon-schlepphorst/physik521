\documentclass[11pt, ngerman, fleqn, DIV=15, headinclude]{scrartcl}

\usepackage[bibatend, color]{../header}

\hypersetup{
    pdftitle=
}

\renewcommand{\thesubsection}{\thesection.\alph{subsection}}

%\subject{}
\title{Statistische Physik, Blatt 1}
%\subtitle{}
\author{
    Frederike Schrödel \and Jan Weber \and Simon Schlepphorst
}

\usepackage{mathtools}

\renewcommand{\thesubsection}{\thesection.\arabic{subsection}}

\begin{document}

\maketitle
\begin{center}
	\begin{tabular}{l|c|c|c}
		Aufgabe &1&2&$\Sigma$\\
		\hline
		Punkte &\quad /10 & \quad /15 & \quad /25 
	\end{tabular}\\
\end{center}


\setcounter{section}{1}

\subsection{Maxwell Relationen und Ableitungsregeln}


\subsection{Wahrscheinlichkeitstheorie}

\subsubsection{ }

\subsubsection{ }

\subsubsection{ }

\subsubsection{ }

\subsubsection{ }

Der Korrelationskoeffizient ist:
\begin{equation}
	K_{ij} := \langle \del{X_i - \langle X_i \rangle} \del{X_j - \langle
	X_j \rangle} \rangle
\end{equation}

\begin{itemize}
	\item  Zweimal geworfener Würfel
		\begin{align*}
			K^1_{12} &= \langle \del{X^1_i - 3.5} \del{X^1_j - 3.5}
			\rangle \\
			&= 0
		\end{align*}
	\item  Zweimaliger Schuss auf Torwand
		\begin{align*}
			K^2_{12} &= \langle \del{0 - 0.5}\del{0 - 0.25} +
			\del{0 - 0.5}\del{1 - 0.25} + \del{1 -0.5}\del{0-0.5} +
			\del{1 -0.5}\del{1 -0.5}\rangle\\
			&= -0.25
		\end{align*}
\end{itemize}

\subsubsection{ }

\subsubsection{ }





\end{document}

% vim: spell spelllang=de tw=79
